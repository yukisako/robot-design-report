\section{はじめに}\label{ux306fux3058ux3081ux306b}

今回の課題は,ロボットが活用されうる状況を自分で想定し,その状況に合わせたロボットを設計することである.その際,ロボットのロボットの必要強度を高めながら手先加速度が大きくなるように設計していくことが目的となる.

今回の課題で私は,食品工場において,クッキーミックスやホットケーキミックスなどの撹拌作業を行うアームロボットを想定し,その水平可動部分の設計を行うこととした.材料を撹拌するために必要な手先速度及び,その手先速度に達するための加速時間の2つから,必要となる手先加速度を定める.最終的にシミュレーションを行い,安全率やたわみ量,手先加速度が条件を満たす適切な形状を見つけていく.
